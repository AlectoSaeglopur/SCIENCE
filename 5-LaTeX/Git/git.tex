
%----------------------------------------------------------------------------------------
%	PACKAGES AND DOCUMENT CONFIGURATIONS
%----------------------------------------------------------------------------------------
\documentclass[a4paper,portrait,10pt]{article}   % set single-column layout
%\documentclass[twoside,twocolumn]{article}	  % set double-column layout
\usepackage{amssymb}
\usepackage[tbtags]{amsmath}

\usepackage{blindtext}   % Package to generate dummy text throughout this template 

\usepackage[sc]{mathpazo}   % Use the Palatino font
\usepackage[T1]{fontenc}   % Use 8-bit encoding that has 256 glyphs
\linespread{1.05}   % Line spacing - Palatino needs more space between lines
\usepackage{microtype}   % Slightly tweak font spacing for aesthetics

\usepackage[english]{babel}   % Language hyphenation and typographical rules

\usepackage[hmarginratio=1:1,top=32mm,columnsep=20pt,left=25mm]{geometry}   % Document margins
\usepackage[hang, small,labelfont=bf,up,textfont=it,up]{caption}   % Custom captions under/above floats in tables or figures
\usepackage{booktabs}   % Horizontal rules in tables

\usepackage{lettrine}   % The lettrine is the first enlarged letter at the beginning of the text

\usepackage{enumitem}   % Customized lists
\setlist[itemize]{noitemsep}   % Make itemize lists more compact

\usepackage{abstract}   % Allows abstract customization
\renewcommand{\abstractnamefont}{\normalfont\bfseries}   % Set the "Abstract" text to bold
\renewcommand{\abstracttextfont}{\normalfont\small\itshape}   % Set the abstract itself to small italic text

\usepackage{titlesec}   % Allows customization of titles
\renewcommand\thesection{\Roman{section}}   % Roman numerals for the sections
\renewcommand\thesubsection{\roman{subsection}}   % roman numerals for subsections
\titleformat{\section}[block]{\large\scshape\centering}{\thesection.}{1em}{}   % Change the look of the section titles
\titleformat{\subsection}[block]{\large}{\thesubsection.}{1em}{}   % Change the look of the section titles

\usepackage{fancyhdr}   % Headers and footers
\pagestyle{fancy}   % All pages have headers and footers
\fancyhead{}   % Blank out the default header
\fancyfoot{}   % Blank out the default footer
\fancyhead[C]{Git Notes $\cdot$ June 2024 }   % Custom header text
\fancyfoot[C]{\thepage}   % Custom footer text
\usepackage{titling}   % Customizing the title section

\usepackage{verbatim}
\usepackage{textcomp}
\usepackage{tikz}
\usetikzlibrary{shapes,arrows}
\usepackage{circuitikz}

%----------------------------------------------------------------------------------------
%	CUSTOM DEFINES
%----------------------------------------------------------------------------------------

\newcommand{\mydiv}{$\rightarrow$ }
%\newcommand{\mydiv}{>> }

\newcommand{\mysapo}[1]{\textquotesingle #1\textquotesingle }   % Single text apostrophe

%----------------------------------------------------------------------------------------
%	TITLE SECTION
%----------------------------------------------------------------------------------------

\setlength{\droptitle}{-4\baselineskip}   % Move the title up
\pretitle{\begin{center}\Huge\bfseries}   % Article title formatting
\posttitle{\end{center}}   % Article title closing formatting
\title{Git Notes}   % Article title
\author{
  \textsc{Filippo Valmori} \\    % Author name
  %\normalsize Alecto S\ae gl\'opur Mul.Dr. \\   % Your institution
}
\date{2\textsuperscript{nd} June 2024}   % Date (leave empty to omit a date)


%----------------------------------------------------------------------------------------
%	DOCUMENT
%----------------------------------------------------------------------------------------

\begin{document}

\tikzstyle{int}=[draw, fill=white, minimum size=2em]
\tikzstyle{init} = [pin edge={to-,thin,black}]
\maketitle   % Print title

%----------------------------------------------------------------------------------
%	INSTALLATION
%----------------------------------------------------------------------------------

\section{Installation}
§ Installation procedure (tested on Windows 10 OS):
\begin{itemize}
  \item[$\cdot$] download and launch installer from Git website (free and open-source);
  \item[$\cdot$] set \textit{C:\textbackslash Program Files\textbackslash Git} as installation path;
  \item[$\cdot$] tick \textit{Open Git Bash here} and untick \textit{Open Git GUI here} within \textit{Windows Explorer integration};
\item[$\cdot$] select \textit{Use Visual Studio Code as Git's default editor};
\item[$\cdot$] select \textit{Let Git decide} about default branch naming;
\item[$\cdot$] select \textit{Git from command line and also from 3rd-party software};
\item[$\cdot$] select \textit{Use bundled OpenSSH};
\item[$\cdot$] select \textit{Use the OpenSSL library};
\item[$\cdot$] select \textit{Checkout Windows-style, commit Unix-style line endings};
\item[$\cdot$] select \textit{Use MinTTY};
\item[$\cdot$] select \textit{Fast-forward or merge} as pull-command behavior;
\item[$\cdot$] select \textit{Git Credential Manager};
\item[$\cdot$] tick \textit{Enable file system caching}
\item[$\cdot$] skip the \textit{Experimental options} window and start the installation.
\end{itemize}

%--------------------------------------------------------------------------
%	SETUP
%--------------------------------------------------------------------------

\section{SETUP}
§ Configure user's name and email:
\begin{itemize}
  \item[$\cdot$] to configure Git username \mydiv \textit{git config -{}-global user.name "Filippo Valmori"};
  \item[$\cdot$] to configure Git email \mydiv \textit{git config -{}-global user.email "filippo.valmori@gmail.com"};
  \item[$\cdot$] NB \#1: for the last x2 commands, \textit{-{}-system} or \textit{-{}-local} could be used in place of \textit{--global} to (however that's in general not recommended, see Coursera's training for more details);
  \item[$\cdot$] to readback set user's name and email \mydiv \textit{git config user.name} and \text{git config user.email} (or all at once via \textit{git config [-{}-global] -{}-list});
  \item[$\cdot$] to avoid line-ending issues among team members working with different OSs and automatically convert \mysapo{CRLF} (typical of Windows) line-endings into \mysapo{LF} (typical of Linux and macOS) when adding a file to the index (and vice versa when it checks out code onto your filesystem) \mydiv \textit{git config -{}-global core.autocrlf true} (in particular, when asserted on Windows machines, this converts \mysapo{LF} endings into \mysapo{CRLF} when you check out code);
  \item[$\cdot$] NB \#2: \mysapo{CR} = \mysapo{\textbackslash r} = \textit{Carriage Return} character | \mysapo{LF} = \mysapo{\textbackslash n} = \textit{Line Feed} character.
\end{itemize}




%--------------------------------------------------------------------------
%	ANNEX
%--------------------------------------------------------------------------

\section{ANNEX}
Hereafter a list of useful trigonometric relations.
\begin{align}
cos^{2}(x) + sin^{2} &= 1 \nonumber \\
2sin(x)cos(x) &= sin(2x) \nonumber \\
1 + cos(2x) &= 2cos^{2}(x) \nonumber \\
cos(x \pm y) &= cos(x)cos(y) \mp sin(x)sin(y) \nonumber \\
sin(x \pm y) &= sin(x)cos(y) \pm cos(x)sin(y) \nonumber
\end{align}

%----------------------------------------------------------------------------------------
%	REFERENCE LIST
%----------------------------------------------------------------------------------------

\begin{thebibliography}{99} % Bibliography - this is intentionally simple in this template

\bibitem[1]{bib:Sklar} B. Sklar, P. K. Ray, \textit{Digital Communications}, Chap. 4-9, Pearson Education, 2012.
 
\end{thebibliography}




% NB#1: MSK come caso di CPFSK [CPFSK]
% NB#2: Schema circuitale da BB a RF [Intro]
% NB#3: Schema circuitale per FSK e GFSK


%----------------------------------------------------------------------------------------

\end{document}
\grid
