%% SET FONT, FORMAT AND TYPE OF DOCUMENT %%
\documentclass[10pt,a4paper]{moderncv}		% Formats: a4paper, letterpaper, a5paper, legalpaper, executivepaper, landscape

%% SET STYLE AND COLOR %%
\moderncvtheme[blue]{classic}				% Color: blue, orange, green, red, purple, grey or black | Style: casual, classic, oldstyle, banking

% ADD PACKAGES %%
\usepackage[italian]{babel}
\usepackage[utf8]{inputenx}
\usepackage[scale=0.828,top=1cm, bottom=2cm]{geometry}

\usepackage{color,soul}						% to use the highlight option (during debug)
\usepackage{fontawesome}					% to use icons in the contacts section 
\moderncvicons{awesome}						% use "awesome" icons

\usepackage{relsize}
\usepackage{xspace}
\usepackage{amssymb, amsmath}
\newcommand{\Rplus}{\protect\hspace{+.01em}\protect\raisebox{.35ex}{\smaller{\smaller\textbf{+}}}}
\newcommand{\Cpp}{\mbox{C\Rplus\Rplus}\xspace}

\setlength{\hintscolumnwidth}{4cm}			% to extend the left column
\AtBeginDocument{\recomputelengths}

\pagenumbering{arabic}						% set page numbering style

\newcommand*{\skypesocialsymbol}{\faSkype~}	% add social symbols
\renewcommand*{\linkedinsocialsymbol}{\faLinkedinSquare}
\RenewDocumentCommand{\social}{O{}O{}m}{%
  \ifthenelse{\equal{#2}{}}{
      \ifthenelse{\equal{#1}{linkedin}}{\collectionadd[linkedin]{socials}{#3}}{}
      \ifthenelse{\equal{#1}{twitter}}{\collectionadd[twitter]{socials}{\protect\httplink[#3]{www.twitter.com/#3}}}{}
      \ifthenelse{\equal{#1}{github}}{\collectionadd[github]{socials}{\protect\httplink[#3]{www.github.com/#3}}}{}
      \ifthenelse{\equal{#1}{skype}}{\collectionadd[skype]{socials}{#3}}{}
    }{\collectionadd[#1]{socials}{\protect\httplink[#3]{#2}}}}


% PERSONAL INFO AND CONTACTS
\firstname{\vspace{.2cm} \hspace{-0.8cm} {\fontsize{35}{35}\selectfont Filippo Valmori}}
\familyname{}
\title{\vspace{1cm} Curriculum Vitae}
%\firstname{\vspace{0.5cm} Filippo Valmori}
%\familyname{}
%\title{\vspace{0.3cm} Curriculum Vitae}
%\address{Via Campo degli Svizzeri 84/b}{Forl\`i (FC), 47121, Italia}
\address{\textbf{\vspace{0.1cm} Contatti}}
%\phone[mobile]{(+39)\,349\,683\,2687}
\phone[fixed]{(+39)\,349\,683\,2687}
\email{filippo.valmori@gmail.com}
\homepage{filippovalmori.wixsite.com/eletlcdsp}
\social[linkedin]{ linkedin.com/in/valmorif}
\social[skype]{filippo.valmori}
\photo[100pt][0.6pt]{Photo.jpg}


%% START OF DOCUMENT BODY %%

\begin{document}
\maketitle
\pagestyle{empty}							% to disable page numbering


\section{Profilo}

\cvline{}{Sono un laureato magistrale in Ingegneria Elettronica e Telecomunicazioni con attualmente pi\`u di 4 anni di esperienza lavorativa in campo aerospaziale, nei quali ho maturato competenze in programmazione firmware, design di circuiti elettrici, signal processing e sistemi a radiofrequenza, oltre a capacità di teamwork e problem solving individuale. Più in generale, sono pronto a imparare ed impegnarmi in progetti scientifici stimolanti e all'avanguardia con ruolo tecnico/R\&D in vista di un percorso di crescita professionale a lungo termine. Inoltre, nel corso degli anni ho sviluppato un sito web personale per l'approfondimento e la condivisione di vari progetti riguardanti elettronica (ELEC), telecomunicazioni (TLC), digital signal processing (DSP) e radio-frequenza (RF).}
\vspace{0.25cm}


\section{Informazioni personali}

\cvline{\textit{nome e cognome}}{Filippo Valmori}
% \cvline{\textit{cognome}}{Valmori}
% \cvline{\textit{sesso}}{M}
\cvline{\textit{data e luogo di nascita}}{24 novembre 1991 | Forl\`i, Italia}
\cvline{\textit{residenza}}{via Ferrante Orselli 32, Forl\`i (FC), 47121, Italia}
\cvline{\textit{cittadinanza}}{italiana}
% \cvline{stato civile}{celibe}
\cvline{\textit{patente di guida}}{cat. B}
\vspace{0.25cm}

\section{Esperienze lavorative}

\cvline{\textit{luglio 2016 - presente}}{\textbf{Ingegnere ELEC \& TLC} presso \textbf{SITAEL S.p.A.} (Forl\`i, Italia) \vspace{0.06cm} \newline
Impegnato nel settore aerospaziale nella progettazione e sviluppo di sistemi elettronici digitali e a radio-frequenza per piattaforme \textit{smallsat}, in particolare con compiti riguardanti:
\begin{itemize}
  \item[--] analisi e previsione della catena di telecomunicazione tra \textit{Spacecraft} (S/C) e \textit{Ground Station} (G/S) in termini di \textit{link budget}, gestione del relativo protocollo (\textit{es.} operazioni di scrambling, codifica, sincronizzazione, modulazione, etc.) per funzionalità di \textit{Telemetry, Tracking and Command} (TT\&C) e richiesta per allocazione frequenziale verso ITU,
  \item[--] programmazione firmware embedded, basata su sistema operativo real-time RTEMS, di microcontrollore STM32 / ARM Cortex-M (dotati di ADC, DMA e interfaccie USART/SPI/CAN integrati), transceiver RF ed altri componenti quali memorie flash, sensori e watchdog,
  \item[--] design e aggiornamento di schematici elettrici e PCB layout tramite software EDA (Altium e OrCAD), calibrazione delle sezioni di protezione (\textit{overvoltage} e \textit{overcurrent}), simulazione circuitale tramite SPICE, definizione di requisiti, analisi di derating e stress termico,
  \item[--] implementazione signal processing della sezione RF/TLC di \textit{Electrical Ground Support Equipment} (EGSE) tramite \textit{Software Defined Radio} (SDR) USRP N210 e software LabVIEW / GNU Radio per la validazione dei requisiti di comunicazione onboard,
  \item[--] stesura di documentazione tecnica per design, test procedure e test report,
  \end{itemize}
}
 
\cvline{}{
\begin{itemize}
  \item[--] svolgimento di campagne di test sia lato S/C che G/S (\textit{es.} test di integrazione HW/SW, test di platform scenario, test RF, verifica di componenti COTS), 	% i.e. overvoltage/current (LCL) check, BER...
  \item[--] supporto per operazioni post-lancio lato G/S con attività di analisi dati e post-processing.
\end{itemize}
}
\vspace{-0.15cm}

\cvline{}{Coinvolto nel corso degli anni in vari programmi in collaborazione con \textit{European Space Agency} (ESA), tra i quali:
\begin{itemize}
  \item[$\cdot$] ESEO, progetto lanciato a dicembre 2018 che ha previsto ampia attività di teamworking con varie università europee,
  \item[$\cdot$] uHETsat, missione di validazione in orbita per thruster a effetto Hall HT-100,
  \item[$\cdot$] SCAT, progetto focalizzato sulla realizzazione di un transceiver in banda C ad alta data-rate (fino a 50 Mb/s con protocollo DVB-S2) basato su FPGA per smallsats,
  \item[$\cdot$] STRIVING, servizio commerciale rivolto ad aziende terze per integrazione e validazione in orbita di payload.
\end{itemize}
}


\cvline{\textit{aprile - giugno 2016}}{\textbf{Ricercatore post-laurea} presso \textbf{Università di Bologna} (Cesena, Italia) \vspace{0.06cm} \newline
Proseguimento e ulteriore sviluppo di tesi magistrale, finanziato da \textit{Consorzio Nazionale Interuniversitario per le Telecomunicazioni} (CNIT), in vista di partecipazione a \textit{International Conference on Ubiquitous Wireless Broadband} (ICUWB) e pubblicazione di articolo scientifico su rivista IEEE [1].}

\vspace{0.25cm}


\section{Titoli di studio}

\cvline{\textit{marzo 2016}}{\textbf{Laurea Magistrale} in \textbf{Ingegneria Elettronica e Telecomunicazioni}%\newline
\vspace{0.06cm}
\begin{itemize}
  \item[--] presso Universit\`a di Bologna (sede di Cesena),
  \item[--] votazione 110/110 con lode,
  \item[--] tesi di laurea sperimentale (in lingua inglese) su reti di sensori radar a banda ultra-larga (UWB) per localizzazione passiva e tracking in ambiente indoor.
\end{itemize}}

\cvline{\textit{ottobre 2013}}{\textbf{Laurea Triennale} in \textbf{Ingegneria Elettronica e Telecomunicazioni}
\vspace{0.06cm}
\begin{itemize}
  \item[--] presso Universit\`a di Bologna (sede di Cesena),
  \item[--] votazione 110/110 con lode,
  \item[--] tesi di laurea sperimentale su \textit{flicker noise} e analisi di affidabilit\`a su dispositivi power-MOSFET sottoposti a stress e successivo recupero tramite processo di annealing.
\end{itemize}}

\cvline{\textit{giugno 2010}}{\textbf{Diploma di Maturità Scientifica}\newline
presso liceo scientifico \textit{Fulcieri Paulucci di Calboli} di Forl\`i.}

\vspace{0.25cm}
%\newpage


\section{Conoscenze linguistiche}

\cvline{\textit{Italiano}}{Madrelingua}
\cvline{\textit{Inglese}}{Avanzato}
\cvline{\textit{Tedesco}}{Base}
%\cvline{\textit{Inglese}}{Conoscenza avanzata (attestato B2), soprattutto in ambito tecnico:
%\begin{itemize}
%  \item[--] capacità di lettura e scrittura: professionale
%  \item[--] capacità orale: buona
%\end{itemize}}
\vspace{0.25cm}


\section{Competenze informatiche e lavorative}

\cvline{\textit{Sistemi operativi}}{Windows, Linux (Ubuntu/Debian), RTEMS.}
\cvline{\textit{Linguaggi software}}{C/\Cpp, Python, Java, VHDL, XML, HTML, LaTeX.}
\cvline{\textit{Software scientifici}}{MATLAB, LabVIEW, GNU Radio, LTspice, Quartus, Altium Designer, OrCAD, CST, AWR, Excel, Eclipse, Git, TortoiseSVN, STK, SpaceCap.}
 % CoDeSys, KiCad
\cvline{\textit{Competenze ulteriori}}{Programmazione di sistemi embedded, DSP e Arduino - Prototipazione FPGA (Intel Cyclone e MAX 10) - Sviluppo di catene di telecomunicazione tramite SDR - Analisi e progetto di front-end a RF - Esperienza nell'utilizzo di strumentazione da laboratorio (\textit{es.} oscilloscopio, analizzatore di spettro, generatore di funzioni) - Programmazione di GUI (Tkinter e MATLAB) - Basi di saldatura - Principi di programmazione PLC - Conoscenze riguardanti antenne, guide d'onda, \textit{electromagnetic compatibility} (EMC), tecniche di modulazione analogiche e digitali, crittografia, codifica di sorgente e canale.}
% Configurazione di reti cablate (\textit{es.} VLAN, router, firewall)
\vspace{0.25cm}


\section{Pubblicazioni}

\cvline{[1]}{F. Valmori, A. Giorgetti, M. Mazzotti, E. Paolini, and M. Chiani, \emph{``Indoor Detection and Tracking of Human Targets with UWB Radar Sensor Networks''}, IEEE International Conference on Ubiquitous Wireless Broadband (ICUWB), Nanjing, China, Oct. 2016}
\vspace{0.25cm}

\section{Partecipazioni}

\cvline{$\cdot$}{ESA Workshop on Aerospace EMC | 20-22 Maggio 2019 | Budapest, Ungheria.}
\cvline{$\cdot$}{STK Comprehensive Training | 5-9 Febbraio 2018 | Mola di Bari, Italia.}
\cvline{$\cdot$}{IEEE European School of Information Theory (ESIT) | 7-11 Maggio 2018 | Bertinoro, Italia.}
\vspace{0.25cm}


\section{Attività extra}

\cvline{$\cdot$}{Sviluppatore di sito web personale per descrizione, implementazione e condivisione di progetti in ambito ELEC, TLC, DSP e RF (quali, ad esempio, 
programmazione di MCU e FPGA, analisi e simulazione circuitale, PCB layout, codifica di canale e modulazioni) $\rightarrow$ \href{https://filippovalmori.wixsite.com/eletlcdsp}{\color{violet} \textit{filippovalmori.wixsite.com/eletlcdsp}}}
\cvline{$\cdot$}{Volantario presso \textit{Croce Rossa Italiana}}
\cvline{$\cdot$}{Appassionato di arte, storia e viaggi}
\cvline{$\cdot$}{Donatore di sangue presso AVIS}


%% BIBLIOGRAPHY %%
%\nocite{*}
%\bibliographystyle{plain}
%\bibliography{nome_file.bib}


%% PRIVACY POLICY AND SIGNATURE%%
\vspace{\fill}
{\footnotesize\noindent
Il sottoscritto è a conoscenza che, ai sensi dell’art. 26 della legge italiana 15/68, le dichiarazioni mendaci, la falsità negli atti e l’uso di
atti falsi sono puniti ai sensi del codice penale e delle leggi speciali. Inoltre, il sottoscritto autorizza al trattamento dei dati personali,
in conformità alle disposizioni della legge sulla privacy (D.L.196/03 e regolamento UE 2016/679).}
\vspace*{0.8cm}\\
Forlì | \today\hfill Firma: \makebox[4cm][b]{\includegraphics[scale=.15]{Signature.jpg}}
\end{document}
